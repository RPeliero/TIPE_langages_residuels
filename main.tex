\documentclass{article}
\usepackage[utf8]{inputenc}
\usepackage{amssymb}
\usepackage{amsmath}


\title{Une question sur les langages rationnels}
\author{}
\date{}

\begin{document}

\maketitle
\tableofcontents


\section{Introduction}

\subsection{Définitions}

\subsubsection{Lettres et mots}

On se donne un ensemble $\Sigma$, fini ou non, qu'on nommera \textit{alphabet}. Ses éléments seront appelés les \textit{lettres}.
\\

On se permet d'utiliser l'étoile de Kleen sur les ensembles :

$$\Sigma^* := \bigcup_{n \in \mathbb{N}} \Sigma^n$$

Où $\Sigma^n$ est l'ensemble des $n$-uplets de lettres, dont on se permettra de noter les éléments abusivement comme suit :

$$ a_1a_2...a_n := (a_1,a_2,...,a_n)$$

On appellera \textit{mot} tout élément de $\Sigma^*$.

\subsubsection{Expressions régulières}

Une \textit{expression régulière} est un mot $f$ de $(\Sigma \cup \{ \cdot \; ; \; | \; ; \; * \: ; \: ( \; ; \; ) \; \})^*$ vérifiant une des propriétés suivantes :

- $f \in \Sigma$

- $f = \phi$

- $f = f_1 \cdot f_2$ où $f_1$ et $f_2$ sont des expressions régulières

- $f = f_1 | f_2$ où $f_1$ et $f_2$ sont des expressions régulières

- $f = (f_0)^*$ où $f_0$ est une expression régulière
\\

Ainsi, " $(a|b)^*\cdot ((b \cdot a) | a) $ " est une expression régulière sur l'alphabet $\{ a ; b\}$
\\
\\

\subsubsection{Langage rationnel}

Soit $f$ une expression régulière, on note $\mathcal{L}(f)$ l'ensemble des mots reconnus (ou dénotés) par $f$. On parlera de \textit{langage rationnel} décrit par $f$. On dira également que $f$ s'interprète en $\mathcal{L}(f)$. 

Ces langages sont définis inductivement comme suit :

- $\mathcal{L}(\phi) = \phi$

- $\forall a \in \Sigma, \mathcal{L}(a) = \{a\}$

- Pour toutes expressions régulières $f_1$ et $f_2$ :
$$\mathcal{L}(f_1 | f_2) = \mathcal{L}(f_1) \cup \mathcal{L}(f_2)$$

- Pour toute expression régulière $f$ : 
$$\mathcal{L}(f^*) = \mathcal{L}(f)^* := \bigcup_{n \in \mathbb{N}} \mathcal{L}(f)^n$$

- Pour toutes expressions régulières $f_1$ et $f_2$ :
$$\mathcal{L}(f_1 \cdot f_2) = \mathcal{L}(f_1) \cdot \mathcal{L}(f_2) := \{ m_1 \cdot m_2 \; ; (m_1,m_2) \in \mathcal{L}(f_1) \times \mathcal{L}(f_2) \}$$


\subsubsection{Résiduel d'un langage}

On appellera \textit{résiduel d'un langage $L$ par rapport à un mot $m$} l'ensemble :

$$m^{-1}L := \{n \in \Sigma^* \; ; \; m \cdot n \in L \}$$

De même, on définit le résiduel d'un langage $L$ par rapport à un langage $M$ comme étant :

$$M^{-1}L := \{ n \in \Sigma^* \; ; \exists m \in M ; \; m \cdot n \in L \}$$

\section{Énoncé du problème}

\subsection{Énoncé}

Le problème que l'on se pose s'énonce ainsi :

$$\boxed{\textup{Caractériser les expressions régulières }f_1,f_2 \textup{ telles que : } \mathcal{L}(f_1)^{-1}\mathcal{L}(f_1 \cdot f_2) = \mathcal{L}(f_2)}$$


\subsection{Remarque}

On sait par avance que, dès lors que $\mathcal{L}(f_1) \neq \varnothing$ :

$$\forall m_2 \in \mathcal{L}(f_2), \forall m_1 \in \mathcal{L}(f_1), m_1 \cdot m_2 \in \mathcal{L}(f_1 \cdot f_2)$$

Autrement dit, quelques soient $f_1$ et $f_2$, on a :

$$\mathcal{L}(f_2) \subset \mathcal{L}(f_1)^{-1}\mathcal{L}(f_1 \cdot f_2)$$

Le problème se pose donc sur l'inclusion réciproque.

\section{Quelques propriétés sur les langages résiduels de l'interprétation d'une expression régulière}

\subsection{Formulaire utile}
On se donne, dans cette partie, deux expression régulières $f_1$ et $f_2$, deux mots $m_1,m_2 \in \Sigma^*$, ainsi que deux lettres distinctes $a,b \in \Sigma$. On a alors les propriétés suivantes.
\\

1) $a^{-1}\mathcal{L}(a) = \varepsilon$
\\

2) $a^{-1}\mathcal{L}(b) = \varnothing$
\\

3) $m_1^{-1}\mathcal{L}(f_1 | f_2) = m_1^{-1}\mathcal{L}(f_1) \cup m_1^{-1}\mathcal{L}(f_2)$
\\

4) Si $\varepsilon \notin \mathcal{L}(f_1)$, alors $a^{-1}\mathcal{L}(f_1 \cdot f_2) = (a^{-1}\mathcal{L}(f_1))\cdot\mathcal{L}(f_2)$
\\

5) Si $\varepsilon \in \mathcal{L}(f_1)$, alors $a^{-1}\mathcal{L}(f_1 \cdot f_2) = (a^{-1}\mathcal{L}(f_1))\cdot\mathcal{L}(f_2) \cup a^{-1}\mathcal{L}(f_2)$
\\

6) $(m_1 \cdot m_2)^{-1}\mathcal{L}(f_1) = m_2^{-1}(m_1^{-1}\mathcal{L}(f_1))$

\subsection{Théorème de Myhill-Nerode}

$$\boxed{\textup{Un langage est reconnaissable si et seulement si il a un nombre fini de résiduels}}$$

\section{Étude de cas particuliers}

Pour commencer, à défaut de résoudre d'emblée le cas général, semblant très compliqué, on va se restreindre à des cas particuliers. Notamment, on se propose de sélectionner des expressions régulières particulières, et de vérifier si elles vérifient l'égalité du problème.

\subsection{Un alphabet restreint}

On suppose dans cette partie que $\Sigma = \{a ; b\}$.

\subsubsection{Cas où $f_1 \in \Sigma$}

On va noter $f_1 = a \in \Sigma$, et $f_2 = f$ dans cette partie

On cherche donc à montrer que :
$$\mathcal{L}(a)^{-1}\mathcal{L}(a \cdot f) = \mathcal{L}(f)$$

Puisqu'une inclusion est évidente, d'après la remarque qui précède, on va se contenter de montrer la seconde :

Soit $m \in \mathcal{L}(a)^{-1}\mathcal{L}(a \cdot f)$. On sait que $a \cdot m \in \mathcal{L}(a \cdot f)$.

Autrement dit :
$$\exists m' \in \mathcal{L}(f) ; a \cdot m = a \cdot m'$$

Ainsi, par régularité, il vient que :

$$\exists m' \in \mathcal{L}(f) ; m = m'$$

D'où $m \in \mathcal{L}(f)$, et l'inclusion cherchée.
\\

Conclusion :
$$\forall f, \forall a \in \Sigma, \boxed{\mathcal{L}(a)^{-1}\mathcal{L}(a \cdot f) = \mathcal{L}(f)}$$
(et ce, quelque soit $\Sigma$)

\subsubsection{Cas où $f_1 = a|b$ et $f_2 = f$}

On se donne deux lettres distinctes $a,b \in \Sigma$, on suppose que $f_1 = a|b$ et on note $f_2 = f$.

On souhaite montrer que :
$$\mathcal{L}(a|b)^{-1}\mathcal{L}((a|b) \cdot f) = \mathcal{L}(f)$$

Tout d'abord, on peut vérifier assez facilement que :

$$\mathcal{L}(a|b)^{-1}\mathcal{L}((a|b) \cdot f) = [\mathcal{L}(a)^{-1}\mathcal{L}((a|b) \cdot f)] \cup [\mathcal{L}(b)^{-1}\mathcal{L}((a|b) \cdot f)]$$

Ainsi, soit $m \in \mathcal{L}(a|b)^{-1}\mathcal{L}((a|b) \cdot f)$, deux cas de figure se présentent :

Ou bien $am \in \mathcal{L}((a|b) \cdot f)$, ou bien $bm \in \mathcal{L}((a|b) \cdot f)$.

Or, de par l'unicité du préfixe de longueur $1$, les deux cas se résument à :

$a \cdot m \in \mathcal{L}(a \cdot f)$ ou $b \cdot m \in \mathcal{L}(b \cdot f)$

Autrement dit, d'après la démonstration du cas précédent, où $f_1 \in \Sigma$, on en déduit que :

$m \in \mathcal{L}(f)$
\\

Conclusion : de par cet argument d'unicité du préfixe de longueur $1$, on peut généraliser le résultat à tout choix de lettres distinctes de l'alphabet :

$$\forall f, \forall n \in \mathbb{N}_{|\Sigma |},\forall a_1,...,a_n \in \Sigma; i \neq j \Rightarrow a_i \neq a_j, $$
$$\boxed{\mathcal{L}(a_1 | ... | a_n)^{-1}\mathcal{L}((a_1 | ... | a_n) \cdot f) = \mathcal{L}(f)}$$


\subsubsection{Un premier cas où l'inclusion non triviale n'est pas vérifiée}

Dans le cas présent, on suppose que $f_1 = (a | (a \cdot b))$ et $f_2 = ((a \cdot a) | (a \cdot b))$

On va se permettre d'omettre le symbole de concaténation, par soucis de lisibilité. On a alors :
$$\mathcal{L}(f_1f_2) = \{ aaa ; aab ; abaa ; abab \}$$

On remarque alors que :
$$m = bab \in \mathcal{L}(f_1)^{-1}\mathcal{L}(f_1f_2)$$

Car $am = abab \in \mathcal{L}(f_1f_2)$

Cependant, $m \notin \mathcal{L}(f_2) = \{aa ; ab \}$


Ainsi, dans ce cas précis :

$$[\mathcal{L}(f_1)^{-1}\mathcal{L}(f_1f_2)]\backslash \mathcal{L}(f_2) \neq \phi$$
\\

\textbf{Généralisation}

Si $f_1$ et $f_2$ sont telles qu'il existe $m_0 , m_1 \in \mathcal{L}(f_1)$, et $r \in \mathcal{L}(f_2)$ tels que :

$$\exists s \in \Sigma* ; [m_1 = m_0s] \wedge [sr \notin \mathcal{L}(f_2)]$$

Alors, puisque $m_1r \in \mathcal{L}(f_1f_2), m_0sr \in \mathcal{L}(f_1f_2)$

Autrement dit :

$$sr \in \mathcal{L}(f_1)^{-1}\mathcal{L}(f_1f_2)$$

Or, comme on l'a supposé, $sr \notin \mathcal{L}(f_2)$
\\

\newpage

\section{Une condition nécessaire}

Soit $\Sigma$ un alphabet fini, $\mathcal{R}(\Sigma)$, l'ensemble des expressions régulières sur $\Sigma$.
\\
\\
On se donne $f_1,f_2 \in \mathcal{R}(\Sigma)$, $\mathcal{A}_1$ reconnaissant $L_1 := \mathcal{L}(f_1)$, et $\mathcal{A}_2$ reconnaissant $L_2 := \mathcal{L}(f_2)$.
\\
On pose $\mathcal{A}_1 = (Q_1, q_{0,1},F_1,\delta_1) \; , \; \mathcal{A}_2 = (Q_2, q_{0,2},F_2,\delta_2)$
\\
On se donne également $\mathcal{A} = (Q, q_{0,1}, F_2, \delta)$, reconnaissant $L_1 \cdot L_2$. (on détaillera sa construction plus tard)
\\
\\
\subsection{Une question intermédiaire : $L^{-1}L$}

Par définition, soit $L \subset \Sigma^*$, on a :
$$L' = L^{-1}L := \{s \in \Sigma^* ; \exists m \in L ; m' = ms \in L\}$$
\\
Soit $s \in L'$, on trouve $m \in L$ tel que $m' = ms \in L$.
\\
En supposant que le langage $L$ soit reconnaissable, alors, on trouve un automate $\mathcal{A} = (Q,q_0,F,\delta)$ reconnaissant $L$.
\\
Puisque $m \in L$, on trouve $q \in F$ tel que $\delta^*(q_0, m) = q$.
\\
De plus, comme $m' \in L$, on trouve $q' \in F$ tel que $\delta^*(q_0,m')=q'$.
\\
Or :
$$q' = \delta^*(q_0,m') = \delta^*(q,s)$$
Autrement dit, on a trouvé pour $s$ deux états finaux $q,q'$ tels que $\delta^*(q,s)=q'$.
\\
Réciproquement, soit $s \in \Sigma^*$, tel qu'il existe $q,q' \in F; \delta^*(q,s)=q'$, alors :

On trouve $m \in L; \delta^*(q_0,m)=q$

On pose $m' = ms$.

Ainsi :
$$\delta^*(q_0,m') = \delta^*(q,s) = q' \in F$$
\\
D'où $m' \in L$, et donc, $s \in L'$.
\\
\\
On a ainsi montré que :

$$\boxed{L^{-1}L = \{ s \in \Sigma^* ; \exists (q,q') \in F' ; \delta^*(q,s) = q' \}}$$
\\

\subsection{Retour au problème}
Soit $\tilde{L} := L_1^{-1}L_1$, l'ensemble des mots constituant des chemins entre deux états finaux de $\mathcal{A}_1$.
\\
Si l'on suppose $\tilde{L}L_2 \neq L_2$, alors, puisque d'évidence, $L_2 \subset \tilde{L}L_2$, on trouve $\tilde{m} \in \tilde{L} \backslash \{\varepsilon\}$ et $m_2 \in L_2$ tel que $\tilde{m}m_2 \notin L_2$.
\\
On trouve $(q_{f,1},q'_{f,1}) \in F_1^2$ tels que :
$$\delta_1(q_{f,1},m)=q'_{f,1}$$
On va également noter $q_{f,2} \in F_2$ tel que :
$$\delta_2^*(q_{0,2},m_2) = q_{f,2}$$
On trouve enfin $m_1 \in L_1$ tel que :
$$\delta_1^*(q_{0,1},m_1) = q_{f,1}$$
\\
Alors, si l'on considère le mot $m_1\tilde{m}m_2$, on a :
$$\delta^*(q_{0,1},m_1am_2) = \delta^*(\delta^*(q_{0,1},m_1),am_2)$$
Ainsi :
$$\delta^*(q_{0,1},m_1\tilde{m}m_2) = \delta^*(q_{f,1},\tilde{m}m_2)$$
D'où :
$$\delta^*(q_{0,1},m_1\tilde{m}m_2) = \delta^*(\delta(q_{f,1},\tilde{m}),m_2)$$
C'est-à-dire :
$$\delta^*(q_{0,1},m_1\tilde{m}m_2) = \delta^*(q'_{f,1},m_2)$$
Et comme $q'_{f,1} \in F_1$, par construction de l'automate $\mathcal{A}$ :
$$\delta^*(q_{0,1},m_1am_2) = q_{f,2} \in F_2$$
\\
Donc le mot $m_1\tilde{m}m_2$ est reconnu par l'automate $\mathcal{A}$.
\\
Autrement dit, $\tilde{m}m_2 \in L_1^{-1}(L_1 \cdot L_2) \backslash L_2$.
\\
\\
On peut donc énoncer une condition nécessaire aux expressions régulières vérifiant l'égalité étudiée :
$$L_1^{-1}(L_1 \cdot L_2) = L_2 \Rightarrow \tilde{L}L_2 = L_2$$
De par le lemme d'Arden, puisque $\varepsilon \in \tilde{L}$, alors on ne peut pas affirmer que $L_2 = \tilde{L}^*$. Cependant, on sait que cette solution à l'équation est la plus petite au sens de l'inclusion. Ainsi, on a : 
$$L_1^{-1}(L_1\cdot L_2) = L_2 \Rightarrow \tilde{L}^* \subset L_2$$
\newpage

\section{Sens réciproque}
On se donne $f_1,f_2 \in \mathcal{R}(\Sigma)$ telles que, pour les notations prises dans la partie précédente :
$$\tilde{L}^* \subset L_2$$
\\
Alors, pour tout mot $\tilde{m} \in \tilde{L}$, on trouve $q,q' \in F_1 ; \delta^*(q,\tilde{m}) = q'$. On notera par la suite : $q \xrightarrow{\tilde{m}} q'$.
\\
\subsection{Passage par les automates généralisés}

D'après le théorème de Kleene (et particulièrement le sens réciproque de l'équivalence qu'il énonce), tout langage reconnaissable est notamment dénoté par une expression régulière.
\\
Pour décrire les langages du problème, et par soucis de clarté dans nos opérations sur les automates, nous nous servirons de la notion d'automate généralisé, qui autorise les transitions à être des expressions régulières, et non plus seulement des lettres. Cela va amplement simplifier les questions d'équivalences entre automates.
\\
La réduction de tels automates s'effectue par la méthode de Brzozowski et McCluskey d'élimination des états.
\\
On peut montrer que l'expressivité des automates généralisés est la même que celle des automates sur $\Sigma$.
\\
Ainsi, il suffit qu'une propriété soit vérifiée sur l'automate généralisé pour que nous revenions aisément à un automate classique.
\\
\\
En l'occurrence, nous allons montrer que dans un cas particulier du sens réciproque, la propriété est vérifiée :

$$\boxed{\textup{\textbf{On suppose par la suite que : }} L_2 \backslash \tilde{L}^* \textup{ \textbf{est reconnaissable}}}$$

(On précise qu'il s'agit ici d'une privation du langage $L_2$ des éléments du langage $\tilde{L}^*$, et non d'un résiduel, comme cela peut parfois être noté dans la littérature à ce sujet)

Alors, $\tilde{L}$ étant fini, on conçoit aisément une expression régulière le dénotant, en sommant la totalité de ses mots. Puis, un passage global de cette expression à l'étoile de Kleene permet d'obtenir un expression dénotant $\tilde{L}^*$. On notera $f^*$ cette expression régulière.
\\
De même, $L_2 \backslash \tilde{L}^*$ étant reconnaissable, on trouve une autre expression régulière le dénotant. On la notera $g$
\\
Enfin, par somme des deux dernières, on obtient une expression régulière dénotant intégralement $L_2$.
\\
On a donc un automate généralisé $\mathcal{A}_{2,\mathcal{G}}$ à deux états : un initial $q_0$, acceptant, et un autre, $q_1$ acceptant. On a ensuite deux transitions :
$$q_0 \overset{f}{\longrightarrow} q_0$$
$$q_0 \overset{g}{\longrightarrow} q_1$$
\\
Puis, par procédé retour, en notant :
$$\tilde{L} = \{ \tilde{m}_i \; ; i \in \mathbb{N}_n\}, \;\;\; n = |\tilde{L}|$$
\\
On a maintenant $n+1$ transitions :
$$\forall i \in \mathbb{N}_n, q_0 \overset{\tilde{m}_i}{\longrightarrow} q_0$$
$$q_0 \overset{g}{\longrightarrow} q_1$$.
\\
Puis, on continue de rebrousser chemin dans l'élimination des états, jusqu'à ce que toute transition ne soit étiquetée que par des mots de $\Sigma^*$ (on ne se permet plus que d'avoir des expressions régulières concaténées entre elles, mais pas des sommes ou des étoiles de Kleene).
\\
\\
On obtient alors un automate $\mathcal{A}_{2,mots}$ d'une toute autre sorte : étiqueté par des mots plutôt que par des lettres. Cependant, cet automate est toujours fini. Et quitte à changer l'alphabet $\Sigma$, puisque le langage qu'il reconnaît est toujours le même (de par les opérations que l'on s'est permis d'effectuer sur les états et les transitions), sa version déterminisée reconnaît également $L_2$.
\\
\section{Une autre approche - à l'aide des dérivées de Brzozowski}

\subsection{Définition}

Soit $a \in \Sigma$, on définit la dérivée de Brzozowski par rapport à $a$ comme suit :

$$\partial_a(\varnothing) = \partial_a(\varepsilon) = \varnothing$$
$$\forall b \in \Sigma, \partial_a(b) = 
\left \{ \begin{array}{cc}
        \varnothing & \textup{ si } b \ne a \\
        \varepsilon & \textup{ sinon }
        \end{array}
\right.$$

$$\partial_a(f|g) = \partial_a(f) | \partial_a(g)$$
$$\partial_a(f^*) = \partial_a(f)\cdot f^*$$
$$\partial_a(f\cdot g) = (\partial_a(f)\cdot g )\;|\;( c(f)\partial_a(g))$$
(où $c(f) = \varepsilon$ si $\varepsilon \in \mathcal{L}(f)$, et $\varnothing$ sinon) 

Puis, de manière récursive, on définit :
$$\forall a \in \Sigma, \forall m \in \Sigma^*, \;\; \partial_{ma} = \partial_a \circ \partial_m$$

Avec notamment $\partial_\varepsilon (f) = f$

\subsection{Une propriété sur les langages résiduels}
$$\mathcal{L}(\partial_m(f)) = m^{-1}\mathcal{L}(f)$$

Ainsi, en notant : $L_1 = \mathcal{L}(f_1)$ et $L_2 = \mathcal{L}(f_2)$, on a

$$L_1^{-1}(L_1 \cdot L_2) = \bigcup_{m_1 \in L_1}m_1^{-1}\mathcal{L}(f_1\cdot f_2)$$

D'où :

$$\boxed{L_1^{-1}(L_1\cdot L_2) = \bigcup_{m_1 \in L_1}\mathcal{L}(\partial_{m_1}(f_1 \cdot f_2))}$$
\\
\\
Or :
$$\boxed{\forall m \in \Sigma^+, \partial_m(f\cdot g) =  \sum_{\underset{uv = m}{u,v \in \Sigma^*}} c \left (\partial_u(f)\right ) \cdot \partial_v(g) }$$

Et en l'occurrence :

$\forall m_1 \in L_1, \forall u,v ; uv = m_1, c(\partial_u(f_1)) = \varepsilon \Leftrightarrow u \in Pref(L_1)\cap L_1$
\\
Or, on sait que $L_1 \subset Pref(L_1)$. Donc $Pref(L_1) \cap L_1 = L_1$.
\\
\\
Et ainsi, les $v$ correspondants complètent les $u$ de $L_1$ en un mot de $L_1$ : $v \in L_1^{-1}L_1 := \tilde{L}$.
\\
\\
Notre équation devient donc :

$$\tilde{L}^{-1}L_2=L_2$$
\\
Or, on le sait, $\varepsilon \in \tilde{L}$. Donc : $L_2 \subset \tilde{L}^{-1}L_2$.
\\
\\
Pour ce qui est de l'inclusion réciproque, on cherche à savoir, pour un $m \in \tilde{L}^{-1}L_2$ quelconque fixé, à montrer qu'il appartient à $L_2$.

Autrement dit, on cherche à montrer que :

$$\forall m \in \tilde{L}^{-1}L_2, \;\;\; c(\partial_m(g)) = \varepsilon$$


À savoir : $$\prod_{m \in \tilde{L}^{-1}L_2} c(\partial_m(g)) = \varepsilon$$

(où la concaténation est notée multiplicativement)

Or, on sait que la fonction $c$ est un morphisme pour la concaténation :

$$c(e\cdot f) = c(e) \cdot c(f)$$

D'où, l'équation à vérifier devient :

$$c \left ( \prod_{m \in \tilde{L}^{-1}L_2} \partial_m(g) \right ) = \varepsilon $$

\subsection{Concaténation de deux dérivées :}

On va chercher une forme plus pratiquable pour la concaténation de deux dérivées d'une même expression par deux mots différents.
\\
Soit $L$ un langage, $m,m' \in L$, $f$ une expression régulière, en notant $L' = \mathcal{L}(f)$ :

On veut une formule pour $\partial_m(f) \cdot \partial_{m'}(f)$

Tout d'abord, en se servant de l'évaluation :

$$\mathcal{L}(\partial_m(f) \cdot \partial_{m'}(f))=\mathcal{L}(\partial_m(f))\cdot\mathcal{L}(\partial_{m'}(f))=m^{-1}L' \cdot m'^{-1}L'$$
















\end{document}
